% to be included by dsmts-userguide.tex

\subsection{dstms--001 --- The birth-death process}

\subsubsection{dstms--001--01} 

This model contains one species, denoted by $X$. The amount of $X$
present in the system is measured in numbers of molecules. The initial
number of molecules of $X$ is 100. The birth reaction 
\[
X \longrightarrow 2X
\] 
has rate parameter \verb$Lambda=0.1$, and the death reaction 
\[
X\longrightarrow \emptyset
\]
has rate parameter \verb$Mu=0.11$. These
rate parameters are global, which means that they apply to the whole
model and not just to a single reaction. Mass-action stochastic
kinetics is assumed. The SBML-shorthand for the model is stored in the
file \verb$dsmts-001-01.mod$, and this is intended to be
human-readable and self-documenting. The SBML file itself is stored in
\verb$dsmts-001-01.xml$, and this is generated automatically from the
corresponding SBML-shorthand. The mean and standard deviation of the
associated Markov process are given in the plots below, and are stored
in the files \verb$dsmts-001-01-mean.csv$ and
\verb$dsmts-001-01-sd.csv$ respectively.

\addplots{dsmts-001-01}

\subsubsection{dsmts--001--02}

This model is the same as \texttt{dsmts-001-01}, except that the rate
parameters \verb$Lambda$ and \verb$Mu$ are declared to be local rather
than global. Therefore the rate parameters apply only to their
respective reactions, but this does not affect the model output
here. 

\addplots{dsmts-001-02}

\subsubsection{dsmts--001--03}

This model is the same as \texttt{dsmts-001-01}, except that the
values of the rate parameters are \verb$Lambda=1$ and
\verb$Mu=1.1$. The skewed distribution associated with this model
means that simulators are likely to fail the suggested standard
deviation test for large values of \verb$t$ (as the normality
assumption underlying the test is clearly invalid in this
case).

\addplots{dsmts-001-03}

\subsubsection{dsmts--001--04}

This is the same as \verb$dsmts-001-01$, except that the initial
number of molecules of \verb$X$ is 10.

\addplots{dsmts-001-04}

\subsubsection{dsmts--001--05}

The same as \verb$dsmts-001-01$, except that the initial number of
molecules of X is 10,000. Due to the large number of molecules
involved, tests using this model will take longer to complete than for
many other models in the suite.

\addplots{dsmts-001-05}

\subsubsection{dsmts--001--06}

The same as \verb$dsmts-001-01$, except that: (i) There is another
species, named \texttt{Sink}, with initial amount 0. The \verb$Sink$
species is declared to be a boundary condition, which means that its
value is not determined by the model reactions but remains constant at
its initial level. (ii) The death reaction is now:
\[
X \longrightarrow Sink 
\]
This model checks that the boundary condition attribute is
handled correctly.

\addplots{dsmts-001-06}

\subsubsection{dsmts--001--07}

The same as \verb$dsmts-001-06$, except that the \texttt{Sink} species
is not a boundary condition. This model is the first one involving two
time-varying species.

\addplots{dsmts-001-07}

\subsubsection{dsmts--001--08}

Same as \verb$dsmts-001-01$, except that the \verb$Cell$ compartment
is declared to have \verb$size=1$. This shouldn't affect
anything.

\addplots{dsmts-001-08}

\subsubsection{dsmts--001--09} 

Same as \verb$dsmts-001-01$, except that the \verb$Cell$ compartment
is declared to have \verb$size=2$. Again, this shouldn't affect
anything, due to the presence of the \verb$hasOnlySubstanceUnits$ flag
in the species declaration.

\addplots{dsmts-001-09}

\subsubsection{dsmts--001--10}

Same as \verb$dsmts-001-01$, except that: (i) the \verb$Cell$
compartment is declared to have \verb$size=1$ (ii)
\verb$hasOnlySubstanceUnits$ is not declared to be true. Although many
simulators will mis-interpret this model, most will get the right
output due to the unit compartmental size. 

\addplots{dsmts-001-10}

\subsubsection{dsmts--001--11}

This is the same as \verb$dsmts-001-01$, except that: (i) the
\verb$Cell$ compartment is declared to have \verb$size=2$ (ii)
\verb$hasOnlySubstanceUnits$ is not declared to be true. Many
simulators (which pass the earlier tests) are expected to fail this
test, due to a lack of sophisticated handling of the
\verb$hasOnlySubstanceUnits$ tag. It would be reasonable for a
simulator to refuse to accept such a model, but a simulator should not
accept the model and then produce incorrect
output.

\addplots{dsmts-001-11}

\subsubsection{dsmts--001--12}

This is the same as \verb$dsmts-001-01$, except that the rate law is
written as \verb$Lambda*X*0.5*2$. This is designed to test the math
expression parsing.

\addplots{dsmts-001-12}

\subsubsection{dsmts--001--13}

This is the same as \verb$dsmts-001-01$, except that \verb$Lambda=0.2$
and the rate law is written as \verb$Lambda*X*0.5$. This is designed
to test the math expression parsing.

\addplots{dsmts-001-13}

\subsubsection{dsmts--001--14}

This is the same as \verb$dsmts-001-01$, except that the rate law is
written as \verb$Lambda*X/2/0.5$. This is designed to test the math
expression parsing.

\addplots{dsmts-001-14}

\subsubsection{dsmts--001--15}

This is the same as \verb$dsmts-001-01$, except that the rate law is
written as \verb$Lambda*(X/2)/0.5$. This is designed to test the math
expression parsing.

\addplots{dsmts-001-15}

\subsubsection{dsmts--001--16}

This is the same as \verb$dsmts-001-01$, except that the rate law is
written as \verb$Lambda*X/(2/2)$. This is designed to test the math
expression parsing.

\addplots{dsmts-001-16}

\subsubsection{dsmts--001--17}

This is the same as \verb$dsmts-001-08$, except that the (unit)
compartmental volume is explicitly included in the rate laws (this
shouldn't affect anything).

\addplots{dsmts-001-17}

\subsubsection{dsmts--001--18}

This is the same as \verb$dsmts-001-17$, except that the
compartmental volume is set to 0.5.

\addplots{dsmts-001-18}

\subsubsection{dsmts--001--19}

This is the same as \verb$dsmts-001-01$, except that 
there is an assignment rule that defines a new species $y$ that is
$2\times X$.

\addplots{dsmts-001-19}



\subsection{dsmts--002 --- The immigration-death process}

\subsubsection{dsmts--002--01}

This model contains one species, denoted by $X$. The amount of $X$
present in the system is measured in numbers of molecules. The initial
number of molecules of $X$ is $0$. The immigration reaction
\[
\emptyset \longrightarrow X
\] 
has global rate parameter
\texttt{Alpha=1}. The death reaction $X\longrightarrow \emptyset$ has
global rate parameter\texttt{Mu=0.1}. Mass-action stochastic kinetics
is assumed.

\addplots{dsmts-002-01}

\subsubsection{dsmts--002--02} 

Same as \verb$dsmts-002-01$,except that
\verb$Alpha=10$. 

\addplots{dsmts-002-02}

\subsubsection{dsmts--002--03} 

Same as \verb$dsmts-002-02$, except the global parameter
\verb$Alpha=10$ is overridden by the local parameter
\verb$Alpha=5$. This is the first model that checks that local
parameters overload global parameters.

\addplots{dsmts-002-03}

\subsubsection{dsmts--002--04}

Same as \verb$dsmts-002-01$, except that \verb$Alpha=1000$. This model
will be slow to run.

\addplots{dsmts-002-04}

\subsubsection{dsmts--002--05} 

Same as \verb$dsmts-002-02$, except that: (i) There are two additional
species: \verb$Source$ and \verb$Sink$. \verb$Source$ and \verb$Sink$
are both boundary conditions, and they both have initial amount
0. (ii) The immigration reaction is now: 
\[
Source \longrightarrow X. 
\]
(iii)The death reaction is now:
\[
X \longrightarrow Sink
\]

\addplots{dsmts-002-05}

\subsubsection{dsmts--002--06} 

Same as \verb$dsmts-002-05$, except that the \verb$Sink$ species is
not a boundary condition. 

\addplots{dsmts-002-06}

\subsubsection{dsmts--002--07} 

Same as \verb$dsmts-002-05$,except that the Sink species is a constant
boundary condition. 

\addplots{dsmts-002-07}

\subsubsection{dsmts--002--08}

Same as \verb$dsmts-002-01$, except that: (i) The global rate
parameter \verb$k$ is set equal to 2. (ii) In the immigration rate,
\verb$k$ is locally set equal to 1(iii) In the death rate, \verb$k$ is
locally set equal to 0.1. This model is designed to test parameter
overloading. 

\addplots{dsmts-002-08}

\subsubsection{dsmts--002--09}

Same as \verb$dsmts-002-01$, except that an event is triggered at time
25 to reset $X$ to a value of 50. This is the first model that
includes an event. It is a timed event.

\addplots{dsmts-002-09}

\subsubsection{dsmts--002--10}

Very similar to \verb$dsmts-002-09$, except that the event is
triggered at time 22.5 and the reset value is 20. The novelty here is
that the event is triggered at a time not corresponding to a time step.

\addplots{dsmts-002-10}

\subsection{dsmts--003 ---Dimerisation }

\subsubsection{dsmts--003--01}

This model contains two species, denoted by \verb$P$ and \verb$P2$,
representing a dimerisation process. The initial numbers of molecules
of P and P2 are 100 and 0 respectively. The dimerization reaction 
\[
2P \longrightarrow P2
\]
has global rate parameter \verb$k1=0.001$. Note
that this is a second order reaction with rate law
\verb$k1*P*(P-1)/2$. The dissociation reaction 
\[
P2 \longrightarrow 2P
\] 
has global rate parameter \verb$k2=0.01$. 

\addplots{dsmts-003-01}

\subsubsection{dsmts--003--02} 

Same as \verb$dsmts-003-01$, except that: (i) the initial number of
molecules of \verb$P$ is 1000 (ii) the values of the rate parameters
are \verb$k1=0.0002$ and \verb$k2=0.004$. 

\addplots{dsmts-003-02}

\subsubsection{dsmts--003--03} 

Same as \verb$dsmts-003-01$, except that in the event that the time
parameter \verb$t$ becomes greater than or equal to 25, the species
populations are reset to the following values: \verb$P := 100$,
\verb$P2 := 0$. This is the first model that tests SBML
\verb$event$s with multiple assignments.

\addplots{dsmts-003-03}

\subsubsection{dsmts--003--04} 

Same as \verb$dsmts-003-01$, except that in the event that the number
of \verb$P2$ molecules becomes greater than 30, the species
populations are reset to the following values: \verb$P := 100$, 
\verb$P2 := 0$. This is the first test of SBML event
handling where the event is triggered by a species --- not a timed event.

\addplots{dsmts-003-04}

\subsubsection{dsmts--003--05} 

Same as \verb$dsmts-003-01$, except that \verb$P$ has been removed
from the model using the conservation law present in the system. This
tests several things, including complex math expression
parsing.

\addplots{dsmts-003-05}

\subsubsection{dsmts--003--06} 

Same as \verb$dsmts-003-05$, except that the rate law is written
slightly differently. This is testing the expression
parser.

\addplots{dsmts-003-06}

\subsubsection{dsmts--003--07} 

Same as \verb$dsmts-003-06$, except that the rate law is written very
slightly differently. This is testing the expression
parser.

\addplots{dsmts-003-07}


\subsection{dsmts--004 --- The batch immigration-death process}

\subsubsection{dsmts--004--01}

This model contains one species, denoted by $X$. The amount of $X$
present in the system is measured in numbers of molecules. The initial
number of molecules of $X$ is $0$. The immigration reaction
\[
\emptyset \longrightarrow 5 X
\] 
has global rate parameter
\texttt{Alpha=1}. The death reaction $X\longrightarrow \emptyset$ has
global rate parameter\texttt{Mu=0.2}. Mass-action stochastic kinetics
is assumed.

\addplots{dsmts-004-01}

\subsubsection{dsmts--004--02} 

Same as \verb$dsmts-004-01$, except that the immigration reaction is 
\[
\emptyset \longrightarrow 10 X
\] 

\addplots{dsmts-004-02}

\subsubsection{dsmts--004--03} 

Same as \verb$dsmts-004-01$, except that the immigration reaction is 
\[
\emptyset \longrightarrow 100 X
\] 

\addplots{dsmts-004-03}


% eof
