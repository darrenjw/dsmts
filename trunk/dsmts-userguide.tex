\documentclass[12pt,a4paper]{article}
\usepackage{graphicx,url,amsmath,hyperref}
\topmargin=-0.5in
\oddsidemargin=0in
\evensidemargin=0in
\textwidth=6.5in
\textheight=9.5in
\newcommand{\addplots}[1]{{\par\includegraphics[width=0.5\textwidth]{#1-mean}\includegraphics[width=0.5\textwidth]{#1-sd}\par}}

\title{SBML DSMTS --- User Guide}
\author{T.~W.~Evans, C.~S.~Gillespie,  \href{http://www.staff.ncl.ac.uk/d.j.wilkinson/}{D.~J.~Wilkinson}\thanks{\texttt{d.j.wilkinson@ncl.ac.uk}}}
\date{\today}

\begin{document}
\sf\maketitle

\section{Introduction} 

This guide describes how the SBML discrete stochastic models test
suite can be used to test a stochastic simulator. Testing a stochastic
simulator is hard, since two identical simulators will never produce
exactly the same set of results. The only sensible method is to run
the simulation lots of times and check that the distribution of
outcomes is correct. This can only be tested in a probabilistic way.
The latest version of this test suite can be obtained from:
\centerline{\url{http://www.calibayes.ncl.ac.uk/Resources/dsmts/}}

\subsection{Aims and Scope of the Test Suite}

The test suite is a set of SBML models each with time course data for
the means and standard deviations of the model species. Developers may
use the suite to check that their simulators produce results that are
consistent with the SBML standard. The test suite assumes that the
simulator produces output on a regular time grid. Of course exact
stochastic simulators naturally produce output on a non-regular grid
corresponding to individual reaction events. However, this ``step
function'' output is easy to map onto a regular time-grid either
post-hoc, or during the simulation run itself. Further discussion of
this issue can be found in Chapter~6 of
Wilkinson~(2006)(\url{http://www.staff.ncl.ac.uk/d.j.wilkinson/smfsb/}),
to which the reader is referred for further discussion of stochastic
simulation generally, and the use of SBML for discrete stochastic
models in particular.

\section{Structure of the Test Suite}

\subsection{Contents of the Test Suite}

The test directory contains the following files for each test model: 
\begin{itemize}
\item \verb$<model>.xml$: the SBML Level 2 test model
\item \verb$<model>-mean.csv$: means of the simulation results at $t = 0,1,\ldots,50$
\item \verb$<model>-sd.csv$: standard deviations of the simulation results at $t = 0,1,\ldots,50$
\item \verb$<model>.mod$: documentation on the model in the form of SBML-shorthand
\item \verb$<model>-mean.pdf$: plot of mean species levels against time 
\item \verb$<model>-sd.pdf$: plot of the standard deviations of the species levels against time. 
\end{itemize}
The first lines of both the \verb$<model>-mean.csv$ and the
\verb$<model>-sd.csv$ files contain the column headings. The first
column heading is \emph{Time}, and the remaining column headings are
the species names in the order they appear in the \verb$listOfSpecies$
in the \verb$<model>.xml$ file. The file \verb$<model>.mod$ contains
the SBML-shorthand for the model
(\url{http://www.staff.ncl.ac.uk/d.j.wilkinson/software/sbml-sh/}),
which is supposed to provide human-readable documentation for each
model. The test suite also contains this documentation, and a file
(\verb$model-list$) containing a simple list of all models in the test
suite (this is useful for batch-testing against all models in the test
suite).

\section{Testing an Exact Simulator}

In order to test the output from a stochastic simulator for a given
SBML model, $n$ independent simulation runs of the simulator should be
performed. The value of $n$ should not be less than 1,000. However,
for the statistical tests to have reasonable power to detect problems,
$n$ should be set to at least 10,000 (but this will be very
time-consuming for some models). The sample means and standard
deviations of the species amounts from the simulation runs at
$t=0,1,\ldots,50$ can be compared with the corresponding values in the
test suite using the statistical tests described below.

\subsection{Testing the simulated mean values}

A test for the simulated values of a species $X$ is as follows: Let
$X_t^{(i)}$ be the value of $X_t$ on the $i^{\mbox{th}}$ run of the
simulator, where $X_t$ is the random variable representing $X$ at time
$t$. Put $\mu_t=\operatorname{E}(X_t)$ and
$\sigma_t=\sqrt{\operatorname{Var}(X_t)}$. By the Central Limit
Theorem, we have
\[
\bar{X_t} \sim N( \mu_t, \sigma_t^2/n ),\ 
\text{ where }
\bar{X_t}=\frac{1}{n}\sum_{i=1}^n X_t^{(i)}.
\]
Therefore
\[
\bar{X_t}-\mu_t \sim N(0,\sigma_t^2/n),
\]
and it follows that
\[
Z_t \equiv \sqrt{n}\left( \frac{\bar{X_t} -\mu_t}{\sigma_t}\right)\sim N(0,1).
\]
So under the null hypothesis that the simulator is correct, the $Z_t$
values should have a standard normal distribution. In this case, most
values will lie in the range (-3,3). Therefore, values of $Z_t$
outside this range correspond to evidence that the simulator is in
error.

\subsection{Testing the simulated standard deviation values}

For this test we need to assume the approximation $X_t\sim
N(\mu_t,\sigma_t^2)$. From this approximation, we have:
\[
\frac{X_t-\mu_t}{\sigma_t} \sim N(0,1) \Rightarrow\frac{1}{\sigma_t^2}(X_t-\mu_t)^2 \sim \chi^2_1,
\]
and so
\[
\frac{1}{\sigma_t^2}\sum_{i=1}^n (X_t^{(i)}-\mu_t)^2 \sim \chi^2_n.
\]
Now put
\[
\hat{S}_t^2 \equiv \frac{1}{n}\sum_{i=1}^n(X_t^{(i)}-\mu_t^{(i)})^2
\]
so that
\[
\frac{n\hat{S}_t^2}{\sigma_t^2} \sim \chi^2_n.
\]
This means that $n\hat{S}_t^2/\sigma_t^2$ will have an approximate
$N(n,2n)$ distribution. Therefore $\hat{S}_t^2/\sigma_t^2$ will have
an approximate $ N(1,2/n)$ distribution. So
\[
Y_t \equiv \sqrt{\frac{n}{2}}\left(\frac{S_t^2}{\sigma_t^2} - 1\right)\sim N(0,1).
\]
Again, values of $Y_t$ outside the expected range for a standard
normal variable correspond to evidence that the simulator is in
error. However, owing to the approximation used in the development of
this test, it is probably best to consider only values outside the
range $(-5,5)$ as evidence of a problem.

\subsection{A note on the tests}

The tests are designed to test simulator accuracy. It is important to
remember that the tests are probabilistic. If lots of tests are
carried out, some failures would be expected to occur by chance, even
if the simulator is behaving correctly. Also, since the standard
deviations test is based on an approximation, the predicted failure
probability for this test is an underestimate. Therefore we might
expect to fail more standard deviation tests than mean tests. Running
the entire test suite with $n=10,000$, it wouldn't be particularly
surprising to fail 2 or 3 mean tests and 5 or 6 standard deviation
tests in total (summed over all models and all time points), even if
the simulator is correct.

\section{Testing an Approximate Simulator}

The test suite is designed first for rigorous testing of exact
simulators. However, it should also prove useful to developers of
approximate simulators. One way of using the test suite to assess the
performance of an approximate simulator is to plot the means and
standard deviations as percentages of their true values. For the mean
values, a plot of $\bar{X_t}/\mu_t$ against $t$ should be produced for
$t=0,1,\ldots,50$ (for $\mu_t\not=0$). Similarly, a plot of
$\hat{S}_t/\sigma_t$ against $t$ should be produced to assess the
standard deviation values ($\sigma_t\not=0$). Ideally, the plots will
be close to 1 for all values of $t$, and values outside the range
$[0.98,1.02]$ (say) is indicative of poor simulator accuracy. For
$\mu_t=0$ or $\sigma_t=0$, a simple check should be made to ensure
that the corresponding sample estimates are (very) close to zero.

\section{Models}

This section lists each model in the test suite, gives a brief
description of it, and provides some indication of what the model is
designed to test. It also includes plots of the means and standard
deviations for each model.

% now import the model descriptions

\input dsmts-models

% now do a final wrapping up?!




\end{document}